\documentclass{beamer}
\usetheme{Madrid}

\title{Kanban e Trello no Contexto Ágil}
\author{Professor Othon Luiz}
\institute{Uni São Miguel}
\date{11/09/224}

\begin{document}

% Slide 1: Título
\begin{frame}
    \titlepage
\end{frame}

% Slide 2: Introdução ao Kanban
\begin{frame}
    \frametitle{Introdução ao Kanban}
    \begin{itemize}
        \item Kanban é um método visual para gerenciamento de fluxo de trabalho.
        \item Foco em melhorar a eficiência e visibilidade do progresso.
        \item Permite controle contínuo e entrega de valor com mais frequência.
    \end{itemize}
\end{frame}

% Slide 3: Princípios do Kanban
\begin{frame}
    \frametitle{Princípios do Kanban}
    \begin{itemize}
        \item Visualize o fluxo de trabalho (quadros e cartões).
        \item Limite o trabalho em andamento (WIP).
        \item Gerencie o fluxo (monitore e ajuste continuamente).
        \item Torne as políticas explícitas (clareza no processo).
        \item Melhoria contínua.
    \end{itemize}
\end{frame}

% Slide 4: Integração com Scrum
\begin{frame}
    \frametitle{Kanban e Scrum}
    \begin{itemize}
        \item Kanban pode ser usado junto com Scrum para gerenciar o fluxo de tarefas.
        \item Scrum:
        \begin{itemize}
            \item Product Backlog: Lista priorizada de tarefas.
            \item Sprint: Ciclos curtos de entrega.
        \end{itemize}
        \item Kanban ajuda a visualizar e otimizar o fluxo.
    \end{itemize}
\end{frame}

% Slide 5: Exemplo no Trello
\begin{frame}
    \frametitle{Exemplo no Trello}
    \begin{itemize}
        \item Trello é uma ferramenta visual para implementar Kanban.
        \item Listas recomendadas:
        \begin{itemize}
            \item Product Backlog
            \item To Do
            \item In Progress
            \item Done
        \end{itemize}
    \end{itemize}
\end{frame}

% Slide 6: Product Backlog
\begin{frame}
    \frametitle{Product Backlog}
    \begin{itemize}
        \item Primeiro quadro à esquerda, usado para armazenar todas as tarefas.
        \item As tarefas são priorizadas e movidas para "To Do" durante o planejamento da Sprint.
    \end{itemize}
\end{frame}

% Slide 7: Aplicação de Limites WIP
\begin{frame}
    \frametitle{Limites de Trabalho em Progresso (WIP)}
    \begin{itemize}
        \item Limitar o trabalho em progresso evita sobrecarga.
        \item Exemplo: Apenas 3 tarefas simultâneas em "In Progress".
        \item Isso mantém o foco da equipe e melhora a eficiência.
    \end{itemize}
\end{frame}

% Slide 8: Exemplo Prático
\begin{frame}
    \frametitle{Exemplo Prático}
    \begin{itemize}
        \item Projeto de desenvolvimento de software:
        \item Product Backlog: Desenvolver página inicial, configurar banco de dados, testar APIs.
        \item To Do: Configurar banco de dados.
        \item In Progress: Desenvolver página inicial.
        \item Done: Testar APIs (concluído).
    \end{itemize}
\end{frame}

% Slide 9: Métricas de Kanban
\begin{frame}
    \frametitle{Métricas de Kanban}
    \begin{itemize}
        \item Lead Time: Tempo desde a criação até a conclusão.
        \item Cycle Time: Tempo para completar uma tarefa após iniciar "In Progress".
        \item Throughput: Número de tarefas concluídas em um período.
    \end{itemize}
\end{frame}

% Slide 10: Conclusão
\begin{frame}
    \frametitle{Conclusão}
    \begin{itemize}
        \item Kanban integrado ao Trello e Scrum melhora a visibilidade e a eficiência do time.
        \item Foco na entrega contínua de valor e adaptação rápida.
        \item Exercício: Criar um board no Trello com as listas do exemplo.
    \end{itemize}
\end{frame}

\end{document}
